\documentclass[twoside]{article}

\usepackage{cite}
\usepackage{listings}
\usepackage{tabularx}

\usepackage[sc]{mathpazo} % Use the Palatino font
\usepackage[OT1]{fontenc} % Use 8-bit encoding that has 256 glyphs
\usepackage[utf8]{inputenc}
\linespread{1.05} % Line spacing - Palatino needs more space between lines
\usepackage{microtype} % Slightly tweak font spacing for aesthetics
\usepackage{graphicx}

\usepackage[hmarginratio=1:1,top=32mm,columnsep=20pt]{geometry} % Document margins
\usepackage{multicol} % Used for the two-column layout of the document
\usepackage[hang, small,labelfont=bf,up,textfont=it,up]{caption} % Custom captions under/above floats in tables or figures
\usepackage{booktabs} % Horizontal rules in tables
\usepackage{float} % Required for tables and figures in the multi-column environment - they need to be placed in specific locations with the [H] (e.g. \begin{table}[H])
\usepackage{hyperref} % For hyperlinks in the PDF

\usepackage{lettrine} % The lettrine is the first enlarged letter at the beginning of the text
\usepackage{paralist} % Used for the compactitem environment which makes bullet points with less space between them

\usepackage{abstract} % Allows abstract customization
\renewcommand{\abstractnamefont}{\normalfont\bfseries} % Set the "Abstract" text to bold
\renewcommand{\abstracttextfont}{\normalfont\small\itshape} % Set the abstract itself to small italic text

\usepackage{titlesec} % Allows customization of titles
\renewcommand\thesection{\Roman{section}} % Roman numerals for the sections
\renewcommand\thesubsection{\Roman{subsection}} % Roman numerals for subsections
\titleformat{\section}[block]{\large\scshape\centering}{\thesection.}{1em}{} % Change the look of the section titles
\titleformat{\subsection}[block]{\large}{\thesubsection.}{1em}{} % Change the look of the section titles

\usepackage{amsmath}

\title{\vspace{-15mm}\fontsize{24pt}{10pt}\selectfont\textbf{Some nice title}}

\date{}

\author{
  \large
  \textsc{Ville Toiviainen}\\[2mm]
  \tt ville.j.toiviainen@aalto.fi \\[2mm]
  \textsc{Antti Partanen}\\[2mm]
  \tt antti.partanen@aalto.fi \\[2mm]
}

%----------------------------------------------------------------------------------------

\begin{document}

\maketitle % Insert title

%----------------------------------------------------------------------------------------
%	ABSTRACT
%----------------------------------------------------------------------------------------

\begin{abstract}
  \noindent{Our plan for this project was to explore neural networks, reinforcement learning and deep reinforcement learning algorithms. We set our focus on two algorithms: Policy Gradient and Deep Q-Networks (DQN), and their use to solve the OpenAI Gym environments. Initially, we selected Tennis on Atari 2600 as the environment to solve, but as we encountered difficulties that we originally didn’t expect, we had to try out our algorithms on other OpenAI Gym’s Atari 2600 environments.

While we didn’t exactly reach our initial goal: solving the Atari Tennis environment, we did manage to create working algorithms of policy gradient and DQN. The algorithms are able to solve the CartPole-v0 environment from OpenAI Gym. In the results section, we discuss the performance of both algorithms.
}


\end{abstract}

%----------------------------------------------------------------------------------------
%	ARTICLE CONTENTS
%----------------------------------------------------------------------------------------

\begin{multicols}{2} % Two-column layout throughout the main article text

  \section{Introduction}
  \label{sec:introduction}
  In recent year biggest development steps in reinforced learning have been related to solutions that use deep learning. The major step was taken by DeepMind team that used Deep Q-Networks with Atari games. For this reason, we wanted to experience with Deep Q-Networks in Atari environments. For comparison, we implemented Policy gradient, which has been an interesting topic of research lately as well.

\subsection{Reinforced Learning}
In reinforced learning, the algorithm learns the environment by interacting with it. It’s a goal-driven way to learn - good actions are often rewarded and bad ones receive a penalty. No wonder that games are a very popular way to test reinforced learning.
To learn the algorithm needs to evaluate its performance - how good decision it was to choose this action? In many cases, the algorithm takes multiple actions before it ends up in the terminal state. This forms multiple distinct action-state sequences that have their own rewards. Markov decision process (MDP) is an important property which means that these sequences can be used as a state. In Q-learning, another important property is that when iteration count approaches to infinity Q-function approaches the optimal Q*. This formalism is called Bellman equation. \cite{mnih2015human}

\subsection{Q-Learning}
In Q-Learning the model learns to estimate the optimal action-value function defined in function \ref{eq:qlearning}.
\begin{equation}
    \label{eq:qlearning}
    Q*(s,a) = \mathbf{E}_{s'} \Big[ r + \gamma \max_{a'} Q*(s',a')|s, a \Big]
\end{equation}
If the environment fulfills the requirements of MDP and Bellman equation eventually when $i \rightarrow \infty$, $Q \rightarrow Q*$.
Q-Learning is an off-policy algorithm which in this case is mandatory since experience replay is used to boost the learning. Experience replay is explained in methods and experiments section. \cite{mnih2015human} \cite{rlintroduction}

\subsection{Deep Learning}
Deep Learning is an area of machine learning that uses deep neural networks for the learning task. ”Neural networks have emerged as an important tool for classification. The recent vast research activities in neural classification have established that neural networks are a promising alternative to various conventional classification methods. The advantage of neural networks lies in the following theoretical aspects. First, neural networks are data-driven self- adaptive methods in that they can adjust themselves to the data without any explicit specification of functional or distributional form for the underlying model. Second, they are universal functional approximators in that neural networks can approximate any function with arbitrary accuracy \cite{zhang2000neural}.” This powerful way to approximate function is especially useful in reinforced learning where a system needs to learn to approximate very complex environment and finding the true optimal function can be too demanding time-wise and computationally.

\subsection{OpenAI Gym}
OpenAI Gym is a toolkit for testing and training reinforcement learning algorithms. OpenAI Gym provides “environments” or test problems that have a generalized interface that allows people to write reinforcement algorithms to try to solve those problems. OpenAI Gym includes different environments, ranging from different games (in 2D and 3D) to various algorithmic problems.

The environments that we tested were Cart Pole, Breakout and Space Invaders.







  \section{Data Analysis}
  \label{sec:methods}
  \input{02-data-analysis.tex}

  \section{Methods and Experiments}
  \label{sec:experiments}
  \input{03-methods-experiments.tex}

  \section{Results}
  \label{sec:results}
  \input{04-results.tex}

  \section{Discussion}
  \label{sec:discussion}
  \input{05-discussion.tex}


  %	REFERENCE LIST
  %----------------------------------------------------------------------------------------

  \bibliographystyle{IEEEtran}
  \bibliography{references}

\end{multicols} % One-column layout for the appendices
\pagebreak
\section{Appendix A: Dataset insights}
\label{appendix-a}
\input{appendix-a.tex}

\section{Appendix B: Pipeline}
\label{appendix-b}
\input{appendix-b.tex}

%----------------------------------------------------------------------------------------

\end{document}

%%% Local Variables:
%%% mode: latex
%%% TeX-master: t
%%% End:
